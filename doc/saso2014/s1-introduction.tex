%!TEX root = main.tex
\section{Introduction}\label{s:intro}

%Our world is filled with examples of communities of biological organisms that exhibit wondrously complex behaviors.  For example, the eusocial {\em Hymenoptera} (sawflies, wasps, bees, and ants) build complex nests that house many thousands of individuals~\cite{grimaldi2005evolution}, while the mounds built by the {\em Macrotermitinae} termites not only house the colony, but also provide thermoregulation and ventilation for the fungus that these termites farm~\cite{korb2003thermoregulation}.  Avian starlings flock to avoid predation~\cite{fernandez-juricic2004flock}, while fish are known to school for a variety of reasons, including protection from predation and reduced energy consumption (see~\cite{stocker1999models} for a brief review).  Finally, biofilms are complex extracellular structures that are formed by nearly all species of microorganism~\cite{davey2000microbial}.

%From large-scale permanent networks such as the Internet to smaller temporary networks such as mobile {\em ad hoc} wireless networks (MANETs), network creation and topology maintenance algorithms are a fundamental aspect of distributed computing systems.  In fixed networks, topology maintenance is typically performed by routers, while in {\em overlay networks}, a logical communication network is created and maintained by application-level software~\cite{jelasity2005t-man}.  In MANETs, the multitude of routing protocols available, and their varying tradeoffs related to reliability and overhead, complicate the engineering of reliable mobile systems~\cite{broch1998a-performance}.  Methods that are able to adapt to the environment in which they are deployed, especially in those systems that interact with physical systems, are needed~\cite{wolf2009cyber-physical}.
%
%Many approaches to solving complex problems in distributed computing have been inspired by observations of biological systems.  For example, Dorigo {\etal}~\cite{dorigo2004evolving} and Baldasarre, Parisi, and Nolfi~\cite{baldassarre2004coordination} have studied coordinated behavior of robotic swarms where individuals were able to physically attach to one another, much like army ants, and the slime mold {\em Physarum polycephalum} has been shown to form networks that are as efficient and fault-tolerant as real-world transportation networks~\cite{tero2010rules}.  Moreover, certain natural behaviors have been codified as generic patterns for use in a wide variety of applications~\cite{babaoglu2006design}.  Yet, determining how these bio-inspired methods can be used to engineer scalable, reliable, and decentralized systems, remains a challenging task.
%
%In this study, we use neuroevolution to discover controllers for a simulated network of mobile autonomous communicating agents, similar to a MANET.  Neuroevolution is a technique whereby an evolutionary algorithm (EA) is used to produce artificial neural networks (ANNs).  Specifically, here we use {\neat}~\cite{stanley2002evolving} to solve a coverage problem, where nodes in a MANET are required to distribute themselves on a grid while maintaining network connectivity.  The ANNs produced by {\neat} were used as controllers for both the movement and communication behavior of nodes in the network, and all nodes in a given network executed a copy of the same evolved ANN.  Nodes were provided with simulated radios, and were able to broadcast to their neighbors within a limited range.  Whether agents were able to communicate with each other was defined by whether a neighbor was transmitting and also by distance between individuals.
%
%The main contribution of this work is to demonstrate a systematic, evolution-based method by which such coverage problems can be addressed, and to show how stability, self-organization, and scalability can be integrated into fitness functions that produce effective solutions for such problems.  These results hint at an evolutionary algorithm-based approach by which large-scale multi-agent control can be achieved.