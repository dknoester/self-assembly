%% To help with graphics importing.
\usepackage{graphicx}
\usepackage{color}
\DeclareGraphicsRule{.pdftex}{pdf}{*}{}
\graphicspath{{figures/}}

% InputFigure
% This command is just a wrapper around \input, to abstract out the path
% and extension of figures.  We're using PDFTeX to make sure that fonts
% are handled correctly from the beginning.
% #1: Filename, without extension or directory.
\newcommand{\InputFigure}[1]{
	\input{figures/#1.pdftex_t}
}

\newcommand{\InputPlot}[1]{
	\includegraphics{#1}
}

%% Some helpful lengths.
\newlength{\HalfPage}
\setlength{\HalfPage}{0.45\textwidth} % <.5, so that we can pack figures side-by-side.
\newlength{\OneColumn}
\setlength{\OneColumn}{0.45\textwidth}

%% New commands; short ones go here.
\newcommand{\dk}[1]{\emph {[[dk: #1]]}}

\newcommand{\mysection}[1]{\section{#1}}
\newcommand{\mysubsection}[1]{\subsection{#1}}

\newcommand{\term}[1]{{\em #1}}
\newcommand{\etal}{{et al.}}
\newcommand{\selfstar}{{\em self-}$\ast$}

\newcommand{\coreworld}{\sc Core~World}
\newcommand{\tierra}{\sc Tierra}
\newcommand{\avida}{A{\sc vida}}
\newcommand{\neat}{NEAT}
\newcommand{\hyperneat}{HyperNEAT}
\newcommand{\multiagent}{Multi-agent HyperNEAT}
\newcommand{\boldavida}{\bf {A\small{VIDA}}}
\newcommand{\instr}[1]{{\sc #1}}
\newcommand{\task}[1]{{\sc #1}}
\newcommand{\event}[1]{{\sc #1}}
\newcommand{\code}[1]{{\sf #1}}

\newcommand{\CalloutAlg}[1]{Algorithm~\ref{#1}}
\newcommand{\CalloutTable}[1]{Table~\ref{#1}}
\newcommand{\CalloutFigure}[1]{Figure~\ref{#1}}
\newcommand{\CalloutChapter}[1]{Chapter~\ref{#1}}
\newcommand{\CalloutSection}[1]{Section~\ref{#1}}
\newcommand{\CalloutEqn}[1]{Equation~\ref{#1}}


%% These are weird...

%%%%%%%%%%%%%%%%%%%%%%%%%%%%%%%%%%%%%%%%%
%
% USAGE: \begin{bfigure}{pos}{wid} text \end{bfigure}
%    pos    the usual figure placement arg: eg. htbp (p.176,latex)
%    wid    the width of the figure, in some units: eg. 5in
%    text   the contents of the figure, including picture/caption/label/etc
%*%  Should steal code from LaTeX to make the interface the same (e.g.
%*%  with `htbp' in braces [] instead of an argument)
\newenvironment{bfigure*}[2]{
    \begin{figure*}[#1]
\centering
\begin{bbox}{#2}
    }{
\end{bbox}
    \end{figure*}
}

%%%%%%%%%%%%%%%%%%%%%%%%%%%%%%%%%%%%%%%%%
% BBOX environment
%%%%%%%%%%%%%%%%%%%%%%%%%%%%%%%%%%%%%%%%%
%
\newenvironment{bbox}[1]{
    \begin{tabular}{|p{#1}|} \hline
}{\\ \hline
    \end{tabular}
}
\newcommand{\spa}{\hspace{1ex}}
\newcommand{\spb}{\hspace{2ex}}


% A ``monkey'' figure function - does it all.
% #1: Width
% #2: Rotate
% #3: Label (filename)
% #4: Caption
\newcommand{\Figure}[4]{
	\begin{figure}[h]
	\center
	\resizebox{#1}{!}{
	\rotatebox{#2}{\includegraphics{figures/#3}}}
	\caption{#4}
	\label{#3}
	\end{figure}
}

\newcommand{\GetFigure}[2]{
	\resizebox{#1}{!}{
	\includegraphics{figures/#2}}
}	

% size
% before skip
% after skip
% figure
% caption
%\newcommand{\FigureSkip}[5]{
%	\begin{figure}[h]
%	\center
%	\vspace{#2}
%	\resizebox{#1}{!}{
%	\includegraphics{figures/#4}}
%	\vspace{#3}
%	\caption{#5}
%	\label{#4}
%	\end{figure}
%}

% Figure start; useful with \SubFloat
%\newcommand{\FigureBegin}{
%	\begin{figure}[htp]
%	\centering
%}

% Figure end; useful with \SubFloat
% #1: Label of whole figure
% #2: Caption of whole figure
%\newcommand{\FigureEnd}[2]{
%	\caption{#2}
%	\label{#1}
%	\end{figure}
%}

% A sub-float; useful with \Figure{Begin/End}
% #1: Width
% #2: Filename
% #3: Caption
%\newcommand{\Subfloat}[3]{
%	\subfloat[#3]{\label{#2}\includegraphics[width=#1]{figures/#2}}
%}

% should eventually save this...
%\floatstyle{boxed}
%\newfloat{program}{thp}{lop}
%\floatname{program}{Program}
%\floatstyle{plain}

%\newcommand{\choose}[2]{\left(\begin{array}{c}#1\\#2\end{array}\right)}

% For including large-ish program fragments.
% #1: Width (not scaled)
% #2: Label (filename)
% #3: Caption
%\newcommand{\Program}[4]{
%	\begin{figure}[htb]
%	\center
%	\lstset{linewidth=#1}
%	\lstset{frame=lrtb}
%	\lstinputlisting[caption=#3,label=#2]{figures/#2}
%	\end{figure}
%}
